%%%%%%%%%%%%%%%%%%%%%%%%%%%%%%%%%%%%%%%%%%%%%%%%%%%%%%%%%%%%%%%%%%%%%%%%%%%%%%%%%%%%%%%%%%%%%%%%%%%%%%%
%%%%%%%%%%%%%% Template de Artigo Adaptado para Trabalho de Diplomação do ICEI %%%%%%%%%%%%%%%%%%%%%%%%
%% codificação UTF-8 - Abntex - Latex -  							     %%
%% Autor:    Fábio Leandro Rodrigues Cordeiro  (fabioleandro@pucminas.br)                            %% 
%% Co-autor: Prof. João Paulo Domingos Silva  e Harison da Silva                                     %%
%% Revisores normas NBR (Padrão PUC Minas): Helenice Rego Cunha e Prof. Theldo Cruz                  %%
%% Versão: 1.0     13 de março 2014                                                                  %%
%%%%%%%%%%%%%%%%%%%%%%%%%%%%%%%%%%%%%%%%%%%%%%%%%%%%%%%%%%%%%%%%%%%%%%%%%%%%%%%%%%%%%%%%%%%%%%%%%%%%%%%
\section{\esp Introdução}

Este artigo apresenta um estudo sobre Obsolescência de Software, em organizações que utilizam Sistemas SAP. A empresa foi fundada em 1972 por cinco ex-funcionários da IBM com o nome SAP – \textit{Systemanalyse und Programmentwicklung} (Análise de Sistemas e Desenvolvimento de Programação em tradução livre do Alemão) com sede em Weinheim, Alemanha. A SAP iniciou suas atividades no ramo de sistemas de processamento real-time, em suas próprias palavras “Software que processa dados quando o cliente solicita, ao invés de passar a noite processando lotes” \cite{sapse}.
\\

Obsolescência é o status dado a um componente que não é mais disponibilizado pelo seu fabricante original. O fabricante descontínua a produção e manutenção por motivos que podem incluir, indisponibilidade de materiais necessários para fabricação, queda na demanda, duplicação de linha de produtos quando organizações se juntam, entre outros. 
A obsolescência de software ocorre quando o desenvolvedor original ou terceiro autorizado para de prover suporte, correções e atualizações regulares, dificultando ou tornam a utilização do Software impossível \cite{rajagopala}

\subsection{\esp Objetivos}

\textbf{Objetivo Geral}\\O objetivo geral deste trabalho é entender a motivação por trás da utilização continuada de software SAP desatualizado ou em situação de obsolescência em organizações brasileiras e apresentar o conceito da obsolescência e DMSMS de Software e sua gerência.

\textbf{Objetivos específicos}\\
Executar uma pesquisa com pessoas da área técnica de desenvolvimento SAP para levantar indicadores e fazer um comparativo sobre dados levantados em 2017 e 2018 com relação a situação encontrada em 2022.
Analisar os impactos em processos, segurança, e experiência do usuário.

\subsection{\esp Motivação}
Trabalhando na área de tecnologia da informação, diariamente temos contato com várias versões de sistemas, muitos deles com anos de defasagem e várias versões mais atualizadas disponíveis, este estudo tem como objetivo executar uma pesquisa visando identificar a motivação por trás da utilização de componentes de software e sistemas SAP obsoletos nos processos das organizações. Analisar os impactos em processos, segurança, custos e posicionamento no mercado.

\subsection{\esp Justificativa}
Em 2015 a SAP apresentou uma nova geração de software empresarial desenvolvida em torno de sua base de dados \textit{in-memory}. A plataforma S/4Hana traz uma mudança histórica na arquitetura do sistema, algo que não ocorria desde o lançamento do ERP R/3, em 1992 \cite{computerworld}. A SAP também anunciou que a versão 6.0 do seu ERP, o ERP Central Component, seria descontinuada em 2025. As versões 5.0 e R/3 do ERP já foram descontinuadas em 2010 e 2016 respectivamente. (SAP PAM, 2021 - Anexo A).
Tendo em vista que os sistemas SAP são utilizados em 33\% das organizações brasileiras e dominando 50\% do segmento das empresas de grande porte \cite{fgv}, surgiu a necessidade de fazer este estudo.


\section{\esp OBSOLESCÊNCIA DE SOFTWARE}

Obsolescência é o status dado a um componente que não é mais disponibilizado pelo seu fabricante original. A fabricante descontinua um componente por motivos incluindo indisponibilidade de materiais necessários para fabricação, queda na demanda, duplicação de linha de produtos quando organizações se juntam. O problema de obsolescência é mais proeminente na tecnologia de eletrônicos, por que, o tempo de aquisição de um componente é significantemente menor que o ciclo de vida de fabricação e suporte dos produtos que utilizam este componente. A obsolescência de produtos se estende para muitas outras áreas como material, têxtil, componentes mecânicos, etc. Inclusive, o problema de obsolescência tem aparecido também no software, especificações, padrões, processos e até mesmo em habilidades de recursos humanos.
\\\\
Obsolescência refere-se a materiais, componentes, dispositivos, software, serviços e processos que ficaram inadquiríveis por meio de seus fornecedores originais. Quando um produto fica obsoleto, seus usuários e clientes ficam em uma situação de escassez de suprimentos quando a demanda por partes ou componentes originais não pode ser atendida pelos fornecedores e não existem opções alternativas disponíveis \cite{COG2005}.

\subsection{\esp Obsolescência involuntária}

A obsolescência involuntária pode ser definida pela seguinte situação: nem o cliente, nem o fornecedor, necessariamente quer que seu produto sofra mudança e se torne obsoleto. Pode-se categorizar em quatro principais categorias: Logística, Funcional, Tecnológica e FIDO \cite{sandborn2012}

\subsubsection{\esp Logística}

Perda da capacidade de se adquirir, materiais, componentes, mão de obra e até mesmo software necessários para a produção, manutenção ou suporte a um produto.
O fornecedor do software não o vende mais como produto novo (\textit{end-of-sale});
A indisponibilidade de se renovar ou adquirir novas licenças (\textit{legally unprocurable} - tradução livre, legalmente incomprável);
Finalização de suporte pelo fornecedor ou terceiros (\textit{end-of-support});

\subsubsection{\esp Funcional}
	
O produto ou subsistema ainda funciona corretamente e ainda pode ser produzido, porém, os requisitos mais específicos do produto mudaram e com isso sua função, performance e confiabilidade ficam obsoletas. Software pode entrar em estado de obsolescência quando hardware, novos requisitos ou outras mudanças de software em um sistema torna a funcionalidade do componente de software obsoleto. O mesmo ocorre quando o hardware não suporta a execução correta de um componente de software que sofreu upgrade, levando-o a obsolescência.

\subsubsection{\esp Tecnológica}
	
Componentes com tecnologia mais avançada estão disponíveis, embora ainda seja adquirível, os sistemas em que estes componentes deveriam integrar não o suportam mais, ou o fornecedor que produz este componente não o suporta mais. Refere-se também como Obsolescência Tecnológica quando, mídias digitais obsoletas, formatos ou degradações acabam por tornar inviável a utilização do software.

\subsubsection{\esp FIDO}
	
\textit{Functionality Improvement Dominated Obsolescence} ou Obsolescência por aperfeiçoamento de funcionalidade. 
Fornecedores são obrigados a evoluir seu produto para manter sua fatia de mercado, FIDO se difere da Obsolescência Funcional pois FIDO é forçada em direção ao fornecedor, enquanto a Funcional é forçada em direção ao consumidor.
\\

A aplicação destas definições pode variar dependendo do sistema em que são usados, e principalmente de onde e como estes sistemas são utilizados. Software de prateleira, tem seus respectivos \textit{end-of-sale} e \textit{end-of-support} separados por longos períodos de tempo. Para muitos softwares de prateleira mais populares estas datas são publicadas por seus fornecedores. Aplicações conectadas ou utilizadas via internet, normalmente tem sua “data de declaração de obsolescência” relacionada a seu \textit{end-of-support}, por ser a data onde terminam as atualizações de segurança e se torna um risco a utilização continuada desde software. Sistemas embarcados ou aplicações isoladas, sua data de obsolescência é dada normalmente pela indisponibilidade de licenças para continuar sua utilização (\textit{legally unprocurable}) ou mudanças nos sistemas em que são embarcados (Obsolescência funcional ou tecnológica).

\subsection{\esp Gerência da Obsolescência}

Para assegurar a sua eficiência, um plano de gerência de obsolescência - PGO - deve ser constantemente aperfeiçoado, o ciclo PDCA (Planejar-Desenvolver-Checar-Agir) desenvolvido pelo Dr. W. Edwards Deming é apropriado para se alcançar isto \cite{sandborn2012}.
Visando suportar o aperfeiçoamento constante, a organização de gerência de obsolescência deve receber os recursos necessários para suas atividades que estão alinhadas com o negócio desta organização. O problema de gerência da obsolescência é muitas vezes referido como DMSMS - \textit{Diminishing Manufacturing Sources and Material Shortages} - que em tradução livre significa, Diminuição de fontes de produção e escassez de materiais - e é definido por Sandborn especificamente para a inabilidade de se adquirir os materiais, componentes ou tecnologias necessárias \cite{sandborn2012}.\\
O PGO para mitigar e evitar os impactos da escassez de todos os tipos de materiais, componentes, dispositivos, software, serviços e processo durante o ciclo de vida de um produto pode ser descrito desta forma, baseado no ciclo PDCA.

\begin{figure}[ht]
	\centering	
	\caption[\hspace{0.1cm}PDCA.]{Ciclo PDCA da Obsolescência}
	\vspace{-0.4cm}
	\includegraphics[width=0.5\textwidth]{figuras/pdca.png}
	% Caption centralizada
	% 	\captionsetup{justification=centering}
	% Caption e fonte 
	\vspace{-0.2cm}
	\\\textbf{\footnotesize Fonte: \cite{sandborn2012} adaptação }	
	\label{fig:figurapdca}
\end{figure}
\vspace{-0.5cm}
\newpage 
Gerência de Obsolescência implica em previsão do ciclo de vida e análises para se identificar os impactos durante todo o ciclo de vida do produto. Os custos relacionados com as ações do PGO devem ser estimados, pessoal deve ser treinado e recursos devem ser adquiridos. Um bom PGO deve ser desenvolvido para assegurar a adequada escolha, o \textit{timing} de implementação e o acompanhamento das ações necessárias.\\

Três categorias de gerência de Obsolescência DMSMS e seus resultados - adaptado e traduzido de \cite{sandborn2008}.


\begin{figure}[ht]
	\centering	
	\caption[\hspace{0.1cm}PGO.]{Diagrama da abordagem do PGO}
	\vspace{-0.4cm}
	\includegraphics[width=0.5\textwidth]{figuras/pgo.png}
	% Caption centralizada
	% 	\captionsetup{justification=centering}
	% Caption e fonte 
	\vspace{-0.2cm}
	\\\textbf{\footnotesize Fonte: \cite{sandborn2008} adaptação }	
	\label{fig:figurapgo}
\end{figure}
\vspace{-0.5cm}

Estas categorias de gerência podem ser descritas da seguinte forma: \cite{sandborn2012}

Gerência Reativa: tem seu foco em determinar uma solução imediata e apropriada para lidar com o problema de um componente que está se tornando obsoleto.\\

Gerência Proativa: implementada para componentes críticos que corram risco de ficar obsoletos.

Gerência Estratégica: significa a utilização de dados de DMSMS, dados logísticos, tecnologias de previsão, e tendências de mercado, otimização de ciclo de vida de produto, para determinar a priori a possibilidade de obsolescência. A abordagem mais comum é o planejamento de renovação, determinando o que deve ser renovado de forma que minimiza os custos no futuro (intercalado com a gerência reativa).

\subsubsection{\esp Custos e gerência}

Existem custos significativos associados a gerência e mitigação de obsolescência de software. As áreas de custo, recursos e tempo devem ser consideradas:

\subsubsection{\esp Mitigação}

Poucas, ou quase nenhuma das abordagens da gerência de obsolescência de hardware são aplicáveis para software. As abordagens mais comuns são License Software \textit{Downgrade}, onde uma negociação com o fornecedor do software que disponibiliza licença da nova versão do software para que seja utilizada na versão obsoleta e \textit{Source Code Purchase} onde literalmente se compra o código fonte do software e a manutenção fica a cargo de terceiros (\textit{third-party escrown}) ou do próprio novo detentor \cite{sandborn2007}. 

\subsubsection{\esp Reimplementação }

Na reimplementação, o software é modificado para que funcione corretamente no novo ambiente, esta abordagem inclui re-teste completo do software e re-arquitetura, em muitos casos também re-integração, migração de dados, novos treinamentos e revisão de documentação \cite{sandborn2007}.

\subsubsection{\esp Requalificação }

Software portado de um ambiente obsoleto para um novo, modificado ou não, deve ser testado novamente e revalidado \cite{sandborn2007}.

\subsubsection{\esp Rehosting }

Rehospedar significa, modificar software existente para operar corretamente em um novo ambiente de desenvolvimento, também chamado de \textit{technology porting}. É aplicável para software legado que foi criado em linguagem e sistemas que ficaram obsoletos. \cite{sandborn2007}.

\subsubsection{\esp Gerência de Mídia }

Armazenamento e manutenção da mídia em que um software está arquivado é um elemento crítico da obsolescência de software. Existem muitos problemas e custos envolvidos que dependem do tipo de mídia, o método de armazenamento e controle de versões \cite{sandborn2007}. 

\subsubsection{\esp Case resolution }

Os custos da  resolução de casos de DMSMS são aplicáveis tanto para software quanto para hardware, custos estes que incluem o acompanhamento de várias métricas de resolução, controle de versão e gerência de base de dados \cite{sandborn2007}. É importante observar que estas abordagens podem ser categorizadas como uma ação de gerência reativa.

\subsubsection{\esp Modelos de previsão}

Herald propõem dois modelos matemáticos para atualização de componentes obsoletos que tem seu foco na otimização dos custos o tempo de vida do sistema:\\
\textbf{\textit{System Element Life Cycle}}\\
O modelo SELCC é baseado em curvas típicas de vendas de produtos, que podem ser usadas para prever o tempo de obsolescência de um elemento do sistema.  \cite{herald2012}.\\
\textbf{\textit{Obsolescence Revision Sequence} (ORS)}\\
O modelo de otimização ORS implementa funções SELCC da perspectiva do sistema, definindo o ciclo operacional e o ciclo de vida do sistema, e uma taxa síncrona de upgrades durante o \textit{life-cycle} de um sistema. \cite{herald2012}.\\

Estes modelos são úteis para situações onde os elementos dos sistemas conseguem trabalhar de forma independente, mudanças otimizadas nos elementos dos sistemas (SEs) podem maximizar a eficiência das operações durante o \textit{life-cycle} desses sistemas. \cite{herald2012}.

\subsection{\esp Obsolescência de software em sistemas SAP }

Tendo em vista que a obsolescência de software é algo que de forma prática não pode ser evitada, como infere Sandborn em seu artigo \cite{sandborn2007}, também é possível identificar este fenômeno nos sistemas SAP utilizados por organizações brasileiras e pela escassez de literatura nacional, é possível também argumentar que a Obsolescência de Software é um tema pouco estudado e difundido no Brasil.

Em uma breve pesquisa exploratória utilizando parte da base de clientes da Engineering do Brasil, foi possível identificar estatísticas iniciais. Foram avaliados inicialmente os sistemas ECC e SRM dos clientes.
O SAP ECC, acrônimo de \textit{ERP Central Component}, é o sistema de planejamento de recursos empresariais que suporta todos os principais processos de negócios, funções e serviços corporativos mais comuns que as empresas necessitam  \cite{BOEDER2014}.

SAP SRM acrônimo de Supplier Relationship Manager que cuida da integração fornecedor/empresa facilitando processos de compras, administração de contratos, e aborda de forma compreensiva a gerência do fluxo de informação entre empresas e seus respectivos fornecedores. \cite{BOEDER2014}

Dos clientes analisados em 2017 e 2018, é possível identificar que 75\% utilizam sistemas SAP com 3 anos até 7 de defasagem, ou seja, sistemas que não receberam atualizações fornecidas diretamente pela SAP. Muitos destes sistemas já possuem 10 novas versões lançadas.

A tabela 1 demonstra os dados relacionados às versões e data de lançamento das versões do componente APPL, relacionado aos módulos de logística e contabilidade do ECC e APPL do SRM, relacionados aos componentes gerência de compra de materiais e serviços, contratos e leilões. Os nomes das empresas foram ocultados por questões de confidencialidade.


% Tabela
\begin{table}[htb]
	\centering
	\caption{\hspace{0.1cm} Empresas e versões de software SAP.}
	\vspace{-0.3cm} % espaço entre titulo e tabela
	\label{tab:tabela1}
	% Conteúdo da tabela
	\begin{tabular}{l|c|c|c|c|c|c}
		\hline
		\textbf{Empresa}	& \textbf{Segmento de Mercado} & \textbf{Software}	& \textbf{Ver.Appl}	& \textbf{Data}& \textbf{APPL. Atual} & \textbf{Data}
		\\
		\hline
		1	& Vacinas e Diagnósticos        & ECC 6.0	& 600.26	& 08.09.2014	& 600.29 & 09.05.2017\\
		2	& Construtora e Engenharia      & ECC 6.0	& 604.08	& 21.10.2010	& 604.19 & 30.05.2017\\
		3	& Construtora e Engenharia      & ECC 6.0	& 604.09	& 01.04.2011	& 604.19 & 30.05.2017\\
		4	& Vacinas e Diagnósticos        & ECC 6.0	& 604.18	& 08.06.2016	& 604.19 & 30.05.2017\\
		5	& Águas e Saneamento            & ECC 6.0	& 604.19	& 30.05.2017	& 604.19 & 30.05.2017\\
		6	& Construtora e Engenharia      & ECC 6.0	& 605.06	& 20.10.2011	& 605.16 & 30.05.2017\\
		7	& Instituição de Ensino         & ECC 6.0	& 605.06	& 20.10.2011	& 605.16 & 30.05.2017\\
		8	& Construtora e Engenharia      & ECC 6.0	& 605.06	& 20.10.2011	& 605.16 & 30.05.2017\\
		9	& Construtora e Engenharia      & SRM 7.0	& 701.07	& 10.02.2012	& 701.16 & 01.06.2017\\
		10	& Peças automotivas             & ECC 6.0	& 617.02	& 16.11.2013	& 617.16 & 20.03.2018\\
		11	& Indústria de Vidro            & ECC 6.0	& 617.02	& 16.11.2013	& 617.16 & 20.03.2018\\
		12	& Biotecnologia                 & ECC 6.0	& 617.05	& 30.04.2014	& 617.16 & 20.03.2018\\

		\hline
	\end{tabular}
	\vspace{.1cm}  %espaço entre tabela e fonte
	\small
	% Fonte
	{\footnotesize\\ \textbf{Fonte: Dados de APPL e data de SAP PAM, 2017-2018}}
\end{table}

Para se interpretar as informações levantadas é preciso conhecer alguns termos utilizados na documentação fornecida pela SAP e os componentes de software envolvidos.\\\\
\textit{Enhancement packages}\\

O SAP ERP foi disponibilizado em 2006 e desde então, funcionalidades adicionais são entregues via pacotes de melhorias, chamados de \textit{SAP Enhancement Packages} (EhP). Estes pacotes permitem que os clientes da SAP, possam gerenciar e disponibilizar novas funcionalidades de software. \cite{SAPERPEHP}.
Para a versão 6.0 do ECC, existem até o momento 7 versões de \textit{Enhancement Packages}, e cada pacote possui seus respectivos \textit{Support Packages}.\\

% Tabela
\begin{table}[htb]
	\centering
	\caption{\hspace{0.1cm} \textit{Enhancement Packages }}
	\vspace{-0.3cm} % espaço entre titulo e tabela
	\label{tab:tabela1}
	% Conteúdo da tabela
	\begin{tabular}{l|c}
		\hline
		\textbf{Versão EHP}	& \textbf{Descrição} \\
		\hline
		600	& Versão do ECC de 2006                               \\
		601	& Primeiro EhP lançado no fim do ano de 2006          \\
		602	& EhP2 publicado em 11 de 2007                        \\
		603	& EhP3 publicado em 5 de 2008                         \\
		604	& EhP4 publicado em 5 de 2009                         \\
		605	& EhP5 publicado em 2010                              \\
		606	& EhP6 publicado em 11 de 2011                        \\
		617	& EhP7 publicado em 2013                              \\
		618	& EhP 8, versão atual lançada em 2016                 \\
		\hline
	\end{tabular}
	\vspace{.1cm}  %espaço entre tabela e fonte
	\small
	% Fonte
	{\footnotesize\\ \textbf{Fonte: Dados de (SAP PAM, 2019)}}
\end{table}
\newpage
Cada um destes pacotes possui correções, melhorias e adequações.

EhP1 - Atualizações e novas funcionalidades no controle de usuários e administração de perfis, nos processos de recursos humanos, finanças, compliance, compras, serviços, etc.

EhP2 - Ajustes e novas funcionalidades foram incluídas nas áreas financeira, Vendas e Serviços, RH, Serviços corporativos, Desenvolvimento e Manutenção de produtos e Logística Executiva.

EhP3 - Incluído o CPE (Commodity Pricing Engine) nos módulos SD (Sales and Distribution) e MM (Material Management).

EhP4 - Inovações no módulo HCM (RH), Finanças, Gerência de Ativos Empresariais, Seguros e Localização para países.

EhP5 - Este pacote introduziu melhorias nas áreas de Vendas, Centrais de Serviços Compartilhados, Compras e Logística Executiva, Gestão de Qualidade (QM), EH\&S ( módulo de Sustentabilidade, Meio-Ambiente, Saúde e Segurança), HCM (RH), Finanças, Gerência de Ativos Empresariais.

EhP6 - Tecnologia HANA e Fiori são adicionadas neste pacote,  junto com 694 funcionalidades. 177 são classificadas como melhorias e novas funcionalidades e o restante melhorias de funcionalidades existentes do pacote anterior.

EhP7 introduziu uma maior adoção do HANA e Fiori para apps, alterando a experiência do usuário se concentrando na simplificação da UX no SAP mobile. EhP7 atua como base para as futuras inovações do SAP Business Suite (UI, SAP HANA) a serem disponibilizadas trimestralmente. Também contém mais três recursos, incluindo uma coleção de aplicativos Fiori, MRP (Planejamento de requisitos de materiais) rodados em HANA e tecnologias para implementação de conceitos de \textit{Data Aging}.

EhP8 oferece inovações e serve de base para a transição para o S/4HANA. O SAP ECC 6.0 será descontinuado em 2025 e o S/4HANA é o novo carro-chefe da SAP no segmento de ERPs, O EhP8 incluir novas funcionalidades executadas através de diferentes indústrias, \textit{Line of Bussiness} e HANA. Além disso, as versões dos componentes válidos para várias aplicações (\textit{cross-application components})  se fundem para simplificar a estrutura.

\section{\esp METODOLOGIA DA PESQUISA}

Desenvolver uma pesquisa é uma tarefa muito parecida com projetar um experimento, pois o projeto deve refletir o objetivo, para que o questionário e análise de suas respostas, esclareça a dúvida proposta. Normalmente uma pesquisa visa alcançar um de dois objetivos. Primeiro caso é tentar descrever um fenômeno de interesse ou no segundo caso, avaliar os impactos de alguma intervenção. \cite{Kitchenham}.

Os autores da publicação \textit{Principles of Survey Research} de 2002, apresentam 2 formas para a construção de uma pesquisa, Design Descritivo e Design Experimental. Baseando na mesma publicação, foi optado pelo Design Experimental e uma de suas 5 opções, Estudos utilizando uma combinação de técnicas.
\\
O processo de design começa pela revisão dos objetivos, examinando a população alvo identificada pelos objetivos, e decidir sobre a melhor forma de obter as informações necessárias para abordar esses objetivos \cite{Kitchenham}.

Utilizando a literatura relacionada a Gerência de Obsolescência de Software, as publicações e documentação de produtos da SAP e seus parceiros, foram desenvolvidas duas pesquisas por meio de questionários visando responder questões relacionadas à Gerência da Obsolescência de Software ou a sua falta, a percepção dos usuários de sistemas SAP em relação a funcionalidade e adequação aos processos, e a percepção dos profissionais envolvidos no desenvolvimento, manutenção e suporte dos sistemas.

\subsection{\esp População da pesquisa }

A população para esta pesquisa deve englobar duas áreas diferentes relacionadas à utilização dos sistemas e engenharia de software. Assim possibilitando a avaliação dos impactos com uma perspectiva de usuário e técnica. Será composta por dois grupos:\\\\
\textbf{Usuários do \textit{SAP Community} - https://community.sap.com\\}
É a rede social para profissionais SAP, possui blogs, wiki, perguntas \& respostas e auxilia no trabalho de milhares de usuários SAP diariamente, com mais de 3 milhões de usuários registrados. (SAP Comunity, 2022).\\\\
\textbf{Grupo ABAP Skype}\\
Grupo no aplicativo de mensagens instantâneas Skype, onde profissionais desenvolvedores SAP e tecnologias relacionadas tiram dúvidas, anunciam oportunidades de emprego e fazem pesquisas. 234 usuários ativos atualmente (Anexo B).\\

\subsection{\esp Técnicas de levantamento de dados }

Serão utilizadas duas técnicas principais de levantamento de dados. Pesquisa por Questionário, aplicada para toda a população de forma direcionada, ou seja, seguindo os três princípios para a efetividade no design de pesquisas \cite{Kitchenham}.\\
1. Resiliência ao viés(\textit{resilient to bias}): Um projeto que não seja indevidamente influenciado por uma determinada facção, aspecto ou opinião. Ou seja, buscar resultados da pesquisa que sejam representativos, reflitam a realidade da situação.\\
2. Apropriado(\textit{appropriate}): Um design que faça sentido no contexto da população. 
Deve ser complexo o suficiente para abordar as questões levantadas pelos objetivos do estudo, e não mais complexo do que precisa ser.\\
3. Rentável(\textit{cost-effective}): Um projeto cuja administração e a análise está dentro dos meios dos recursos alocados à pesquisa. Esta relação custo-eficácia aplica-se à pesquisa participantes também; os resultados da pesquisa devem ser tão úteis para eles que vale a pena o tempo para completar a pesquisa.

Entrevistas com indivíduos da população definida que ao responder os questionários demonstram interesse em uma conversa detalhada e ampla sobre o assunto, visando obter informações de como é a visão dos profissionais da área técnica e gerencial sobre o assunto.

\section{\esp AVALIAÇÃO DOS RESULTADOS}

Analisando a pesquisa feita com grupo de desenvolvedores SAP em 2017 foi possível encontrar indicadores:
Segundo a pesquisa, quando questionados sobre o conhecimento do termo “Gerência de Obsolescência de Software“ apenas 23,9\% dos entrevistados possuíam conhecimento anterior sobre o termo, porém em 95,7\% dos casos não é um tema discutido no ambiente das empresas em que trabalham, como pode ser visto na \cite{fig:figura3}.

\begin{figure}[ht]
	\centering	
	\caption[\hspace{0.1cm}Q1.]{Conhecimento do tema}
	\vspace{-0.4cm}
	\includegraphics[width=0.8\textwidth]{figuras/pergunta_1_e_2.jpg}
	% Caption centralizada
	% 	\captionsetup{justification=centering}
	% Caption e fonte 
	\vspace{-0.2cm}
	\\\textbf{\footnotesize Fonte: Pesquisa de opinião Software Obsoleto 2017 e 2018 }	
	\label{fig:figura3}
\end{figure}
\vspace{-0.5cm}

78,3\% dos desenvolvedores confirmam que tem contato com sistemas SAP Obsoletos ou com versões desatualizadas durante a rotina de trabalho, enquanto 17,4 informa que não. 4,3 não souberam opinar.

\begin{figure}[ht]
	\centering	
	\caption[\hspace{0.1cm}Q1.]{Contato com software obsoleto\/desatualizado}
	\vspace{-0.4cm}
	\includegraphics[width=0.6\textwidth]{figuras/pergunta-4.png}
	% Caption centralizada
	% 	\captionsetup{justification=centering}
	% Caption e fonte 
	\vspace{-0.2cm}
	\\\textbf{\footnotesize Fonte: Pesquisa de opinião Software Obsoleto 2017 e 2018 }	
	\label{fig:figura4}
\end{figure}
\vspace{-0.5cm}
\newpage
Apenas 34,8\% das respostas apresentaram contato com as versões mais atualizadas
do ERP, 618 ou com o ERP novo lançado em 2015 S/4HANA. É possível verificar que 50\% das respostas incluem a versão 617, o que pode indicar
que as empresas estavam no caminho para atualizar seu Sistema ERP para a
plataforma S/4HANA, tendo em vista que a versão 618 é o ultimo passo para o upgrade.
A versão 617 foi lançada no ano de 2013, o que pode ser um indicativo de quase 10 anos de defasagem.  

\begin{figure}[ht]
	\centering	
	\caption[\hspace{0.1cm}Q1.]{Versões do ERP}
	\vspace{-0.4cm}
	\includegraphics[width=0.8\textwidth]{figuras/pergunta-5.png}
	% Caption centralizada
	% 	\captionsetup{justification=centering}
	% Caption e fonte 
	\vspace{-0.2cm}
	\\\textbf{\footnotesize Fonte: Pesquisa de opinião Software Obsoleto 2017 e 2018 }	
	\label{fig:figura5}
\end{figure}
\vspace{-0.5cm}
\newpage
A incompatibilidade de soluções já existentes nas versões mais novas com as versões obsoletas e mais antigas é a maior ocorrência, nestes casos a solução deve ser reimplementada.A dificuldade de reaproveitamento de código também é apontada como um problema recorrente, seguido com a obrigatoriedade de utilização de funções já em desuso (\textit{deprecated}) e problemas de performance nos ambientes.

\begin{figure}[ht]
	\centering	
	\caption[\hspace{0.1cm}Q1.]{Erros encontrados}
	\vspace{-0.4cm}
	\includegraphics[width=0.8\textwidth]{figuras/pergunta-erros.png}
	% Caption centralizada
	% 	\captionsetup{justification=centering}
	% Caption e fonte 
	\vspace{-0.2cm}
	\\\textbf{\footnotesize Fonte: Pesquisa de opinião Software Obsoleto 2017 e 2018 }	
	\label{fig:figura6}
\end{figure}

Para os casos de sistemas sem suporte, como os primeiros \textit{enhancement packages}, quando não existe opção de atualização pelas ferramentas da própria SAP,
68,4\& das respostas indicam que a solução deve ser desenvolvida de forma independente, quase 37\% acabam por terceirizar a manutenção ou atualização. 31,6\% dos casos indicam uma resolução utilizando softwares auxiliares e middlewares. 

\begin{figure}[ht]
	\centering	
	\caption[\hspace{0.1cm}Q1.]{Como as organizações lidam com a obsolescência}
	\vspace{-0.4cm}
	\includegraphics[width=0.8\textwidth]{figuras/pergunta-update.png}
	% Caption centralizada
	% 	\captionsetup{justification=centering}
	% Caption e fonte 
	\vspace{-0.2cm}
	\\\textbf{\footnotesize Fonte: Pesquisa de opinião Software Obsoleto 2017 e 2018 }	
	\label{fig:figura7}
\end{figure}
\vspace{-0.5cm}

\subsection{\esp Conclusão}

TODO:
